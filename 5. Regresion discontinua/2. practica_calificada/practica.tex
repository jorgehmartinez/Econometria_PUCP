\documentclass{article}
\usepackage[utf8]{inputenc}
\usepackage[spanish]{babel}
\usepackage{amsmath}
\usepackage{amsfonts}
\usepackage{amssymb}

\setlength{\parindent}{0in} %Evita sangría en cada párrafo
\usepackage[margin=2cm]{geometry} % Margenes 

% Header y footer
\usepackage{fancyhdr} 
\pagestyle{fancy}
\fancyhead[L]{Pontificia Universidad Católica del Perú}   
\fancyhead[R]{QLAB}  

 % OPCIONES DE FORMATO DE PÁGINA:
\textwidth=450pt \textheight=620pt \oddsidemargin=0in
\topmargin=-.2in

\begin{document}

\begin{center}


\LARGE{\textbf{Práctica Calificada: Regression Discontinuity}}\\

Profesor: Tomás Rau

\end{center}

Esta tarea podrá ser respondidas en grupos de hasta 5 personas. Se pide elaborar un informe con las respuestas y adjuntar el código en STATA o R.

\bigskip

La base de datos \textbf{almond.dta} es una submuestra de los datos utilizados en el paper \textit{Estimating Marginal Returns to Medical Care: Evidence from At-risk Newborns} de Almond, Doyle, Kowalski \& Williams (2010); de modo que muchos de los resultados que replique no serán exactos, sino más bien parecidos.

\begin{enumerate}
\item[a)] Lea el paper y conteste las siguientes preguntas:
	\begin{enumerate}
	\item ¿Por qué podría haber una discontinuidad en el estado de salud de los recién nacidos justo debajo vs justo encima de 1500gr?
	\item ¿Cuál es el supuesto fundamental que se hace sobre la salud de los recién nacidos cerca del corte \textit{Very Low Birth Weight} que permite a los autores estimar el retorno marginal de los cuidados médicos?
	\item ¿Qué podemos decir sobre el supuesto a medida que nos alejamos (por ambos lados) del peso de corte?
	\item ¿Cuáles son las 2 formas en las que los autores miden inputs sobre el estado de salud?
	\item ¿De qué forma miden el output o resultado sobre el estado de salud?
	\item ¿Es un diseño Sharp o Fuzzy? Justifique. 
	\end{enumerate}
\item[b)] Análisis preliminar: los autores proponen 2 formas de estimar la discontinuidad del outcome en 1500gr: una \textit{Local Linear Regression} y un modelo lineal controlando por covariates y con efectos fijos. Comencemos con el modelo lineal. Para todas las preguntas, ignore y omita los efectos fijos y los covariates de la ecuación 1 del paper.
	\begin{enumerate}
	\item Escriba la forma funcional del valor esperado de $Y$ para recién nacidos con un peso de (i) 1450, (ii) 1550, (iii) 1499 y (iv) 1501 gramos. Nota: no necesita estimar nada ni utilizar datos
	\item ¿Cómo interpreta los coeficientes $\alpha_2$ y $\alpha_3$, y por qué podrían ser distintos?
	\item En el punto de corte hay 2 interceptos, uno para aquellos justo debajo de la clasificación VLBW y el otro para los recién nacidos justo encima. Escriba la forma funcional de ambos interceptos e interprete la diferencia (i.e. la resta de ambos interceptos).  
	\end{enumerate}

\item[c)] Análisis preliminar 2: Piense ahora en la \textit{Local Linear Regression}:
	\begin{enumerate}
	\item ¿Cuál es la intuición de una regresión Local?
	\item ¿Por qué es importante la elección del ancho de banda? Explique el trade-off sesgo-varianza.
	\item ¿Cómo elegir entonces un ancho de banda óptimo? Explique brevemente los métodos propuestos por Imbens and Kalyanaraman (2012) y Calonico, Cattaneo, and Titiunik (2015).
	\item ¿Qué rol juega la elección del Kernel? Explique brevemente la diferencia entre Kernel Uniforme, Triangular y Epanechnikov.
	\end{enumerate}

\item[d)] Estime el ATE en el punto de corte utilizando el modelo lineal, interprete sus resultados, compárelos con los obtenidos por los autores y construya una gráfica de la esperanza condicional de $Y$ en el peso al nacer. Nota: Incluya la nube de puntos (scatter) en su gráfico.  

\item[e)] Estime el ATE en el punto de corte utilizando un \textit{Local Linear Regression} (comando \textit{rdrobust}). Pruebe con distintas especificaciones:
	\begin{itemize}
	\item Para un ancho de banda dado (CCT), estime con Kernel Uniforme, Triangular y Epanechnikov. Comente sus resultados
	\item Para un Kernel dado (Epanechnikov), estime con CCT, IK, 2*CCT y 0.5*CCT. Comente sus resultados y compárelos con los obtenidos por Almond et al (2010).
	\end{itemize}

\item[f)] Utilice el comando \textit{rdplot} para graficar la relación no lineal entre la esperanza condicional de $Y$ en el peso al nacer y compare con la gráfica construida en d).

\item[g)] Construya el histograma del peso al nacer, realice el test de McCrary y discuta sus resultados. ¿Cuál es el argumento de los autores para justificar que la discontinuidad de la densidad no representa una amenaza a la identificación? Nota: debe instalar manualmente el test en Stata desde la página web del autor.

\item[h)] Lea el paper de Barreca, Guldi, Lindo \& Waddell (2011) y explique brevemente el argumento de los autores para revisitar los resultados de Almond et al (2010).

\item[i)] Barreca et al (2011) proponen una estimación alternativa conocida como \textit{Donut-hole RD}. Estime el ATE en el punto de corte con una \textit{Local Linear Regression}, (i) eliminando las observaciones con peso de 1500 y (ii) eliminando las observaciones con peso entre 1499 y 1501. Comente sus resultados y compárelos con los obtenidos por Barreca et al (2011). 
\end{enumerate}
%\end{enumerate}

\end{document}